% !TEX root = ../main.tex
%---------------------------------------------------------------------------------------------------
\section*{Motivation}
%---------------------------------------------------------------------------------------------------
Whenever computational models inform decision-making uncertainties are ubiquitous. These models use data to inform decisions and my be very sensitive to the uncertainty associated with the parameters estimated from the data. A statistical decision generalizes the concept of computational models determining the outcome of any action or policy. They map the realization of data to a set of possible policies or actions. The performance of each statistical decision function is then determined by the true but unknown parametrization of the model. \\

For an evaluation of different decision functions, the performance under each possible true parametrization needs to be assessed. A naive evaluation implies high computational costs, which are often not feasible due to time or technical constraints. The application of different decision criteria, which guide the choice of a desirable decision function, require to evaluate the performance for some parametrizations very accurate and can be relaxed for others. This allows to use sparse or adaptive grid methods, to reduce the computational burden. \\

This handout first presents a theoretical framework for the use of statistical decision criteria. A simple urn example demonstrates the application to statistical decision functions. Throughout we compare, the classic approach of as-if decision making, i.e. where the statistical decision functions is solely constructed with the point-estimates of the parameters and robust statistical decision functions, which use the realized data as well as some correction to provide robust performance across possible parametrizations.
